
\chapter{Завдання \theHchapter}

\begin{tcolorbox}[title=Завдання]
    Доведiть, що якщо функцiя $f$ зростає на $[0, 1]$, то полiном 
    Бернштейна $B_n(f, \cdot)$ теж зростає на $[0, 1]$.
\end{tcolorbox}

\center{\bfseries Розв'язання:}

нехай $f$ зростаюча на $[0, 1]$ функція 


$B_{n +k}(f, x) = \sum\limits_{r = 0}^{n+k}f(\frac{r}{n+k})
C_{n+k}^r x^r(1 - x)^{n+k-r}$


Запишемо загальний вигляд k-ї похідної


$B_{n +k}^{(k)}(f, x) = \sum\limits_{r = 0}^{n+k}
f(\frac{r}{n+k})C_{n+k}^r p(x),
\quad p(x) = \frac{d^k}{dx^k}x^r(1 - x)^{n + k -r}$


За правилом Лейбніца 


$$
\cfrac{d^s}{dx^s} \ x^r = 
\begin{cases}
    \frac{r!}{(r-s)!}x^{r-s}, r - s \geq 0 \\
    0, r-s < 0    
\end{cases}
$$ 
$$
\qquad
\cfrac{d^{k - s}}{dx^{k -s}} \ (1 - x)^{n+k-r} = 
\begin{cases}
    (-1)^{k + s}\frac{(n+k-r)!}{(n+s-r)!}(1 - x)^{n+s-r}, r - s \le n \\
    0, r-s > n    
\end{cases}
$$

\newpage
Тоді отримаємо формулу


$$
p(s) = \sum\limits_s (-1)^{k-s} C_k^s \frac{r!(n+k-r)!}{(r-s)!(n+s-r)!}
x^{r-s}(1-x)^{n+s-r}
$$


Після заміни індексів $t = r-s$ можна перетворити суми
$
\sum\limits_{r=0}^{n+k}\sum\limits_{s} = 
\sum\limits_{t=0}^{n}\sum\limits_{s=0}^{k}$.


Використавши допоміжну теорему з курсу функціонального аналізу отримаємо


$$
B_{n+k}^{(k)}=\frac{(n+k) !}{n !} \sum_{r=0}^n \Delta^k f\left(\frac{r}{n+k}\right)\left(\begin{array}{c}
n \\
r
\end{array}\right) x^r(1-x)^{n-r}
$$
(крок $\Delta = \frac{1}{n+k}$ )


В нашому випадку $k = 1$


$$ B_{n +1}^{(1)} = \frac{(n+1) !}{n !} \sum_{r=0}^n \Delta f
\left(\frac{r}{n+1}\right)
\left
(\begin{array}{c}n \\
r
\end{array}
\right) x^r(1-x)^{n-r} $$


$\Delta f$ додатня атже функцiя зростаюча, отже всі множники додатні і 
похідна також $\Rightarrow$ поліном зростає.


Доведено!