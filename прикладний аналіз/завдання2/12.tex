
\chapter{Завдання \theHchapter}

\begin{tcolorbox}[title=Завдання]
    Пропонується узагальнити теорему Стоуна для 
    локально компактного простору $X$. 
    Нагадаємо, що простiр $X$ локально компактний, якщо кожна точка 
    $X$ має компактний окiл.
    Позначимо $C_{\infty}(X)$ — лiнiйний простiр неперервних функцiй 
    $X \rightarrow \mathbb{R}$, що зникають у нескiнченностi, 
    тобто $f \in C_{\infty}(X)$ мають властивiсть:
    
    
    $ \forall \epsilon > 0 \quad \exists$ компакт $K \subseteq X: |f(X)| < \epsilon, 
    \forall x \notin K $


    Простiр $C_{\infty}(X)$ з рiвномiрною нормою є банаховим.
    Доведiть таку теорему.
\end{tcolorbox}


\begin{tcolorbox}[title=Теорема Cтоуна для локально компактного простору.]
    Нехай $X$ — локально компактний простiр, $A \subseteq C_{\infty}(X)$. 
    
    
    Припустимо, що:


    1) $A$ — алгебра;

    
    2) $A$ роздiляє точки множини $X$;
    
    
    3) $\forall x \in X \quad \exists f \in A: f(x) \neq 0$.
    
    
    Тодi множина $A$ щiльна в $C_{\infty}(X)$.
\end{tcolorbox}

\center{\bfseries Розв'язання:}