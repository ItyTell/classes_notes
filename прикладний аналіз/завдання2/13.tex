
\chapter{Завдання \theHchapter}

\begin{tcolorbox}[title=Завдання]
    Розглянемо комплексний банаховий простiр 
    $C_{\mathbb{C}}(X)$ заданих на компактi X неперервних
    функцiй $X \rightarrow C$ (норма рiвномiрна). 
    Для комплексних алгебр $A \subseteq C_{\mathbb{C}}$ 
    теорема Стоуна не вiрна (чому?). 
    Але цю ситуацiю можна виправити, якщо вимагати вiд алгебри $A$ 
    ще одну умову. Доведiть таку теорему.
\end{tcolorbox}


\begin{tcolorbox}[title=Теорема Cтоуна для комплексних алгебр]
    Нехай $X$ — компакт, $A \subseteq C_{\mathbb{C}}$. 
    Припустимо, що:


    1) $A$ — алгебра;


    2) $A$ роздiляє точки множини $X$;


    3) $\forall x \in X \quad \exists f \in A: f(x) \neq 0$;


    4) $\forall f \in A$ функцiя $f$, визначена рiвнiстю
    
    
    \quad $f(x) = f(x), x \in X$, належить $A$.


    Тодi множина $A$ щiльна в $C_{\mathbb{C}}(X)$, 
    тобто $clA = C_{\mathbb{C}}(X)$.
\end{tcolorbox}

\center{\bfseries Розв'язання:}