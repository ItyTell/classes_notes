
\chapter{Завдання \theHchapter}

\begin{tcolorbox}[title=Теорема Л.Фейєра]
    Нехай $f \in C_{2 \pi}, S_n$ - часткова сума ряду Фур'є функції $f$ по основній тригонометричній системі. Тоді послідовність середніх Чезаро
    $$
    \frac{S_0+S_1+\ldots+S_{n-1}}{n}
    $$
    рівномірно на $\mathbb{R}$ збігається до $f$.

\end{tcolorbox}

\center{\bfseries Розв'язання:}


Підставляючи коефіцієнти Фур'є у формули для 
часткових сум одержуються загальні формули:

$s_n(x)=\sum\limits_{k=-n}^n(\frac{1}{2 \pi} \int\limits_{-\pi}^\pi 
f(t) e^{-i k t}d t) e^{i k x}=\frac{1}{2 \pi} \int\limits_{-\pi}^\pi 
f(t) \sum\limits_{k=-n}^n e^{i k(x-t)} d t=$


$=\frac{1}{2 \pi} \int\limits_{-\pi}^\pi f(t) D_n(x-t) dt$


Після заміни змінних можна також написати


$$s_n(x)=\frac{1}{2 \pi} \int\limits_{-\pi}^\pi f(x-t) D_n(t) d t$$


де $D_n(t)$ позначає відповідне ядро Діріхле.
Тоді також


$$\sigma_n(x)=\frac{1}{n+1} \sum\limits_{k=0}^n s_k(x)=\frac{1}{2 \pi} 
\int\limits_{-\pi}^\pi f(x-t)(\frac{1}{n+1} \sum\limits_{k=0}^n D_n(t)) 
d t=\frac{1}{2 \pi} \int\limits_{-\pi}^\pi f(x-t) F_n(t) d t$$


де $F_n(t)$ позначає відповідне ядро феєра.
Далі, враховуючи, що для всіх ядер Феєра 
$\frac{1}{2 \pi} \int\limits_{-\pi}^\pi F_n(t) d t=1$, також можна записати


$$\sigma_n(x)-f(x)=\frac{1}{2 \pi} 
\int\limits_{-\pi}^\pi(f(x-t)-f(x)) F_n(t) d t$$


Оскільки ядро Феєра є невід'ємною функцією, то звідси:


$$|\sigma_n(x)-f(x)| \leqslant \frac{1}{2 \pi} 
\int\limits_{-\pi}^\pi|f(x-t)-f(x)| F_n(t) d t$$


Оскільки $f \epsilon$ неперервною на проміжку $[-\pi, \pi]$, то вона $€$ 
на ньому рівномірно неперервною, тобто для кожного $\varepsilon>0$ існує 
$\delta>0$ таке, що для всіх $|x-y| \leqslant \delta
|f(x)-(y)|<\varepsilon / 2$. 
Інтеграл із останньої рівності можна записати як суму 
$\frac{1}{2 \pi} \int\limits_{-\pi}^\pi
|f(x-t)-f(x)| F_n(t) d t=I_1+I_2$, де:


$$I_1=\frac{1}{2 \pi} \int\limits_{|t| \leqslant \delta}
|f(x-t)-f(x)| F_n(t) d t$$


$$I_2=\frac{1}{2 \pi} \int\limits_{\pi \geqslant|t| \geqslant \delta}
|f(x-t)-f(x)| F_n(t) d t$$


Через рівномірну неперервність функції $f$ і знову використавши рівність $\frac{1}{2 \pi} \int_{-\pi}^\pi F_n(t) d t=1$, для першого інтегралу
$$
I_1<\frac{1}{2 \pi} \int_{|t| \leqslant \delta} \varepsilon F_n(t) d t=\varepsilon
$$
Для другого інтегралу, якщо позначити $M=\sup _{x \in[-\pi, \pi]}|f(x)|$, то
$$
I_2 \leqslant \frac{1}{2 \pi} \int_{\pi \geqslant|t| \geqslant \delta} 2 M F_n(t) d t=\frac{M}{\pi} \int_{\pi \geqslant|t| \geqslant \delta} F_n(t) d t
$$
Згідно властивостей ядра Феєра останній вираз прямує до нуля для великих $n$, тобто для достатньо великих $n$ :
$$
I_2 \leqslant \frac{M}{\pi} \int_{\pi \geqslant|t| \geqslant \delta} F_n(t) d t<\varepsilon / 2
$$
Остаточно у цьому випадку
$$
\left|\sigma_n(x)-f(x)\right| \leqslant \frac{1}{2 \pi} \int_{-\pi}^\pi|f(x-t)-f(x)| F_n(t) d t=I_1+I_2<\varepsilon / 2+\varepsilon / 2=\varepsilon .
$$
Тобто $\sigma_n(x)$ прямує до $f(x)$ і крім того збіжність є рівномірною оскільки індекс $n$ у доведенні вище був обраний єдиним для всіх $x$.


Доведено!

