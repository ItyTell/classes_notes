
\chapter{Завдання \theHchapter}

\begin{tcolorbox}[title=Означення 1]
    Нехай $S \subseteq C(X)$. 
    Множина $S$ сильно pозділяє точки множини $X$, якщо
    $\forall x_1, x_2 \in X, x_1 \neq x_2, \forall a_1, a_2 \in \mathbb{R},
    \exists f \in S: f(x_1) = a_1, f(x_2) = a_2$
\end{tcolorbox}

\begin{tcolorbox}[title=Означення 2]
    Нехай $S \subseteq C(X)$. Множину $S$ називають решіткою, 
    якщо $\forall f, g \in S$:
    $\max\{f,g\} \in S, \min\{f,g\} \in S$.
\end{tcolorbox}


\begin{tcolorbox}[title=теорема Какутанi–Крейна.]
    Нехай $X$ — компакт, $S \subseteq C(X)$. Припустимо, що:

    \begin{enumerate}[1)]
        \item S — решітка
        \item S — замкнена множина 
        \item S сильно розділяє точки множини Х
        \item $1 \in S$
    \end{enumerate}

    
    S співпадає з yciм простором $C(X)$.
\end{tcolorbox}

\center{\bfseries Розв'язання:}


\pagebreak

Нехай $h \in C(X)$ і дана $\varepsilon$. 


Ми шукаємо $f \in S$ що задовільняє умову 
$\|h-f\|<\varepsilon$.


Покажемо для кожного $x \in X$, існує 
$f_x \in S$ така, що 


$f_x(x)=h(x)$ і $h \leq f_x+\varepsilon$.


Тоді для кожного $x$, знайдемо $U_x$, відкритий окіл $x$ з 
$h(y) \geq f_x(y)-\varepsilon$ 


для кожного $y \in U_x$ (з неперервності $h-f_x$ ). 


$U_x$ покриття $X$ тож нехай $U_{x_1}, \ldots, U_{x_n}$ підпокриття. 


Тоді $f=f_{x_1} \wedge \cdots \wedge f_{x_m}$ задовільняють умову


$f(y)+\varepsilon=$ $\min _i\left\{f_{x_i}(y)+\varepsilon\right\} 
\geq h(y)$. 


Більше того з того, що $y \in U_{x_i}$ для певного $i$:


$f(y)-\varepsilon$ $\leq f_{x_i}(y)-\varepsilon \leq h(y) \le 
f_{x_i} + \varepsilon$. 


Таким чином $\|f-h\|_{\infty}<\varepsilon$.

Тепер спробуємо знайти $f_x$, що задовільняють цим умовам. 


З того, що $S$ сильно розділяє точки і $1 \in S$, 
для кожного $x$ та $y$ в $X$, ми можемо знайти $f_{x y} \in S$ 
з $f_{x y}(x)= h(x)$ і $f_{x y}(y)=h(y)$. 


Для кожного $y$, ми можемо знайти $V_y$, відкриту множину навколо $y$ 
з $f_{\dot{x}_y}(z)+\varepsilon \geq h(z)$ 
для $z \in V_y . V_{y_1}, \ldots, V_{y_n}$ є покриттям $X$ 
для підходящих $y_1, \ldots, y_n$. 


Якщо взяти $f_x=f_{x y_1} \vee \cdots \vee f_{x y_n}$, 
тоді $f_x(x)=h(x)$, і для будь-якого $z \in X$
$$
f_x(z)+\varepsilon=\max _{i=1, \ldots, n}\left\{f_{x y_1}(z)+\varepsilon\right\} \geq h(z)
$$


Це завершує доведення