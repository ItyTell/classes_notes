
\chapter{Завдання \theHchapter}

\begin{tcolorbox}[title=Означення 1]
    Нехай $S \subseteq C(X)$. 
    Множина $S$ сильно pозділяє точки множини $X$, якщо
    $\forall x_1, x_2 \in X, x_1 \neq x_2, \forall a_1, a_2 \in \mathbb{R},
    \exists f \in S: f(x_1) = a_1, f(x_2) = a_2$
\end{tcolorbox}

\begin{tcolorbox}[title=Означення 2]
    Нехай $S \subseteq C(X)$. Множину $S$ називають решіткою, 
    якщо $\forall f, g \in S$:
    $\max\{f,g\} \in S, \min\{f,g\} \in S$.
\end{tcolorbox}


\begin{tcolorbox}[title=теорема Какутанi–Крейна.]
    Нехай $X$ — компакт, $S \subseteq C(X)$. Припустимо, що:


    1) S — решітка;
    
    
    2) S — замкнена підмножина $C(X)$;
    
    
    3) S сильно розділяє точки множини Х;
    
    
    4) $1 \in S$
    
    
    S співпадає з yciм простором $C(X)$.
\end{tcolorbox}

\center{\bfseries Розв'язання:}