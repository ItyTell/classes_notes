
\chapter{Завдання \theHchapter}


\begin{tcolorbox}[title=Завдання]
    
    Нехай C — непорожня опукла пiдмножина гiльбертового 
    простору $ H, \dim H < + \infty, x \in H \backslash C $.
    Доведiть, що $ \exists p \in H \backslash \{0\} $ такий, що
    $$ \s (p, y) \le (p, x) $$

    
\end{tcolorbox}

\center{\bfseries Розв'язання:}


З завдання 4 випливає що якщо $ x \in H \backslash \bar{C}, 
\exists p \in H\backslash \{0\} \quad $
$ \s (p, y) < (p, x) $

Залишилось довести твердження для $x \in \bar{C} \backslash C$

Так само візьмемо за $ p = x - P_Cx $

$ (p, y) - (p, x) = (x - P_Cx, y) - (x - P_Cx,x) = $


$=(x - P_Cx, y - P_Cx) + (x - P_Cx, x - P_Cx) = 
(x - P_Cx, y - P_Cx) + \| x - P_Cx \|^2$


Очевидно, що для $x \in \bar{C} \backslash C, P_Cx = x $
(є послідовність з $C$ що збігається до $x$, і тоді норма різниці буде 0)  

$(x - P_Cx, y - P_Cx) + \| x - P_Cx \|^2 = (0, y - x) = 0 $


$ 
\begin{cases}
    \s (p, y) < (p, x), \quad \forall x \in C \\
    \s (p, y) = (p, x), \quad \forall x \in \bar{C} \backslash C
\end{cases} $

Отже 


$ \s (p, y) \le (p, x) , \quad \forall x \in C $ 


Доведено