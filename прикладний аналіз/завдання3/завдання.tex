\documentclass[10pt]{article}
\usepackage[utf8]{inputenc}
\usepackage[T1]{fontenc}
\usepackage{amsmath}
\usepackage{amsfonts}
\usepackage{amssymb}
\usepackage[version=4]{mhchem}
\usepackage{stmaryrd}
\usepackage{bbold}

\title{Завдання 3. Навколо теорем Брауера та Какутані }

\author{}
\date{}


\begin{document}
\maketitle
\begin{enumerate}
  \item Нехай $A_{1}, A_{2}, A_{3}$ - набір замкнених підмножин трикутника $\Delta \subseteq \mathbb{R}^{2}$ з вершинами $v_{1}, v_{2}$ та $v_{3}$. Нехай:
\end{enumerate}

\begin{enumerate}
  \item $\Delta=\bigcup_{k=1}^{3} A_{k}$;

  \item $\forall k \in\{1,2,3\}: v_{k} \in A_{k}$;

  \item $\forall k, i \in\{1,2,3\}:\left[v_{k}, v_{i}\right] \subseteq A_{k} \cup A_{i}$.

\end{enumerate}

Доведіть, що $\bigcap_{k=1}^{3} A_{k} \neq \emptyset$.

\begin{enumerate}
  \setcounter{enumi}{1}
  \item Доведіть, що всі опуклі компакти з непорожньою внутрішністю в $\mathbb{R}^{n}$ гомеоморфні.

  \item Нехай $A \subseteq B^{n}$ - непорожня замкнена множина. Доведіть, що існує неперервне відображення $T: B^{n} \rightarrow B^{n}$ таке, що $F(T)=A$, де $F(T)-$ множина нерухомих точок відображення $T$.

  \item Нехай неперервне відображення $f: B^{n} \rightarrow \mathbb{R}^{n}$ має властивість:

\end{enumerate}

$$
(f(x), x) \geq 0 \quad \forall x \in S^{n-1} .
$$

Доведіть, що існує точка $x_{0} \in B^{n}: f\left(x_{0}\right)=0$.

\begin{enumerate}
  \setcounter{enumi}{4}
  \item Нехай неперервне відображення $f: B^{n} \rightarrow \mathbb{R}^{n}$ має властивість:
\end{enumerate}

$$
f\left(S^{n-1}\right) \subseteq B^{n}
$$

Доведіть, що існує точка $x_{0} \in B^{n}: f\left(x_{0}\right)=x_{0}$.

\begin{enumerate}
  \setcounter{enumi}{5}
  \item В банахових просторах $\ell_{2}, c_{0}$ та $C([-1,1])$ побудувати приклади неперервних відображень, що відображають замкнену кулю в себе, але не мають нерухомих точок.

  \item Нехай $H$ - нескінченновимірний гільбертовий простір. Доведіть, що оператор проектування на замкнену кулю не є слабко неперервним.

  \item Нехай $\left(X, d_{X}\right),\left(Y, d_{Y}\right)-$ метричні простори, $\left(Y, d_{Y}\right)-$ компактний простір. Нехай $f \in$ $C(X \times Y)$ та $g(x)=\max _{y \in Y} f(x, y)$. Доведіть, що $g \in C(X)$.

  \item Нехай функція $\phi: X \times Y \rightarrow \mathbb{R}$ неперервна, $Y-$ компакт. Доведіть, що відображення $T: X \rightarrow 2^{Y}$, задане співвідношенням

\end{enumerate}

$$
T x=\left\{\bar{y} \in Y: \phi(x, \bar{y})=\inf _{y \in Y} \phi(x, y)\right\}
$$

замкнене ${ }^{1}$.

\begin{enumerate}
  \setcounter{enumi}{9}
  \item Нехай $A, B$ - непорожні опуклі компакти з банахових просторів $X, Y$, відповідно. Функція $L: X \times Y \rightarrow \mathbb{R}-$ неперервна на $A \times B$ та опукла по $x$ на $A$ (для всіх $y \in B$ ), угнута по $y$ на $B$ (для всіх $x \in A$ ). Доведіть, що існує сідлова точка функції $L$ на $A \times B$, тобто, існує $\left(x_{0}, y_{0}\right) \in A \times B$
\end{enumerate}

$$
L\left(x_{0}, y\right) \leq L\left(x_{0}, y_{0}\right) \leq L\left(x, y_{0}\right) \quad \forall x \in A \forall y \in B
$$

$1_{X, Y-\text { метричні простори. }}$


\end{document}