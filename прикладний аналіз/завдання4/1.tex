
\chapter{Завдання \theHchapter}

\begin{tcolorbox}[title=Завдання]
    Покажіть, що лема 1 з лекції 7 справедлива для строго 
    опуклих лінійних нормованих просторів.
\end{tcolorbox}

\begin{tcolorbox}[title=Лема]
    Нехай $H$ — строго опуклий лінійний нормований простір, 
    $C \subseteq H$ — опукла замкнена
    множина, $T : C \rightarrow H$ — нерозтягуючий оператор. 
    Тодi множина $F(T)$ опукла та замкнена.
\end{tcolorbox}

\center{\bfseries Розв'язання:}


З неперервності $T$ випливає замкненість $F(T)$


Якщо повторити доведення з лекції у кінці вийде $\|\dots \| \le 0$ замість $=0$, 
тобто рзультат залишиться тим самим $F(T)$ - опукла.


Доведено!