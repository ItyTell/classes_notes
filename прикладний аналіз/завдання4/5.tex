
\chapter{Завдання \theHchapter}

\begin{tcolorbox}[title=Завдання]
    Нехай $H$ - гільбертовий простір, $T: H \rightarrow H-$ твердо нерозтягуючий
    оператор із $F(T) \neq \emptyset$. Розглянемо метод простої ітерації:
    $$
    \left\{\begin{array}{l}
    x_{0} \in H \\
    x_{n+1}=T x_{n}
    \end{array}\right.
    $$

    Доведіть, що послідовність $x_{n}$ слабко збігається 
    до деякої точки з $F(T)$
\end{tcolorbox}

\center{\bfseries Розв'язання:}


Оператор нерозтагуючий, послідовність обмежена тоді будуємо множини


$C_n = \bigcap\limits_{k=n}^{\infty}B_d(x_k), d = diam(x_n)$


Множини мають властивість $T(C_n) \subseteq C_{n+1}$ (з побудови послідовності)


$C = cl\bigcup\limits_{n=1}^{\infty}C_n$ - замкнена, опукла та обмежена і за теоремою
Браудера (з лекції 7) оператор $T$ має нерухому точку в $C$.


Доведено!
