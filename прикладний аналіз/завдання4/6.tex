
\chapter{Завдання \theHchapter}

\begin{tcolorbox}[title=Завдання]
    В банахових просторах $\ell_{2}, c_{0}$ та $C([-1,1])$ 
    побудувати приклади неперервних відображень, 
    що відображають замкнену кулю в себе, але не мають нерухомих точок.

\end{tcolorbox}

\center{\bfseries Розв'язання:}


це не можливо за теоремою Брауера, адеж подібне відображження завжди матиме
непорожню множину нерухомих точок.