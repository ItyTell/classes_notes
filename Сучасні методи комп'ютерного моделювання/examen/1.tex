
\chapter{Завдання \theHchapter}

\begin{tcolorbox}[title=Завдання 12]
    Задачі лінійного, нелінійного та стохастичного програмування.
\end{tcolorbox}



Задача у перших двох полягає у мінімізації(максимізації) певної функції 
на певному просторі: $\min\limits_{x \in X} f(x)$.


Почнемо з лінійної оптимізації. Функції у задачах лінійного програмування 
лінійні, тобто переводять лінійні системи у лінійні системи. 
Простір пошуку $X$ - опуклий многогранник утворений за допомогою перетинів 
гіперплощин (лінійні обмеження), також можна записати умову $Ax \le b$. 
У данній оптимізації розв'язок шукають в одній із вершин многогранника.
(за основною теоремою з лінійного програмування). Будь-яка так задача 
розв'язується симплекс методом або його більш ситуантивними підвидами 
(метод потенціалів для транспортної задачі і тд.).


Коли або функція нелінійна або умови нелінійні то це вже задача нелінійного 
програмування. 
Якщо лише умові нелінійні то часто таку задачу можна звести до подібної 
задачі лінійного програмування.
Якщо функція нелінійна уже постає питання про існування
розв'язку як такого, тож часто додають умову сильної опуклості і уточнюють 
що пошук ведеться на компакті. Такі задачі називають задачами опуклої 
оптимізації.
Іноді замість цього вводять поняття псевдорозв'язку з меншою нормаю і тд. 
Хоча частіше шукають будь-який розвязок навіть якщо він виявиться 
не єдиним.


У цих двох підвидах математичної оптимізації ми по суті можемо керувати 
всім задля знаходження розв'язку. Якщо ж є деякі стохастичні (некеровані) 
параметри від яких також залежить і цільова функція $f$ то це вже 
стохастична оптимізація.
Тут навіть може мінятись сама постановка задачи - можуть не мінімізувати 
саму функцію, а її ймовірнісної характеристики (матсподіювання, мода і тд). 
