
\chapter{Завдання \theHchapter}

\begin{tcolorbox}[title=Завдання 13]
    Достатні умови асимптотичної збіжності ітераційних алгоритмів.
\end{tcolorbox}


Перш за все для збіжності треба існування самого розв'язку на заданій 
множині(було б добре мати сильно опуклу функцію на компакті).

Ітераційний алгоитм буде асимптотично збіжний, якщо $\forall \varepsilon >0$
$\exists N \in N - $ номер ітерації такий, що $x_N$ на данній ітерації 
потрапляє в $O_{\varepsilon}(x^*)$


Основні умови для ітераційних алгоритмів потрібні для доведення збіжності 
це умови на крок.
Якщо ітераційний метод має змінний крок $\alpha_s > 0$ то варто 
до нього застосувати такі умови:


$\lim\limits_{s \rightarrow \infty}\alpha_s = 0$ - при наближенні 
до розв'язку (збільшені кількості ітерацій) крок має зменшуватись 
інакше можливе зациклювання.


$\sum\limits_{s = 1}^{\infty} \alpha_s = \infty$ - умова завдяки, якій крок 
не стане занадто малим 


Для градієнтних методів, слід зазначити, що антиградієнт при наближені 
до розв'язку зменшується і тому на ітераціях добуток кроку на антиградієнт 
теж зменшується, але при великому незмінному кроці бувають зациклювання 
тож ці умови є необхідними.