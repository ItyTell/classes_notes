\section*{Розв'язання}
\begin{tcolorbox}[title=Завдлання 1]
    1. Перевірити чи є вказаний функціонал $f$ на просторі $E$ лінійним та неперервним. У випадку лінійного функціоналу обчислити його норму.

    $
    1.1) f(x)=\sum_{k=1}^{\infty} \frac{x_{k}}{k(k+1)}, E=c_{0}, x=(x_{1}, x_{2}, \ldots) 
    $

    $$ f(ax+by) = \sum_{k=1}^{\infty} \frac{(ax_{k} + by_{k})}{k(k+1)} = 
    \sum_{k=1}^{\infty} a\frac{x_{k}}{k(k+1)} + 
    \sum_{k=1}^{\infty} b\frac{y_{k}}{k(k+1)} = af(x) + bf(y)$$

    $$ f(x) = \sum_{k=1}^{\infty}\frac{x_{k}}{k(k+1)}= |x\in c_0|= f(x) \le 
    \sum_{k=1}^{\infty}\frac{x_{0}}{k(k+1)} = x_0\sum_{k=1}^{\infty}\frac{1}{k(k+1)} 
    = x_0$$ 

    $$\text{де } x_0 = \max_{x_k \in x} x_k = \|x\| 
    \Rightarrow f(x) \le \|x\| \Rightarrow \|f\| = 1$$

    $
    1.2) f(x)=\sum_{k=1}^{\infty} \frac{x_{k}}{k}, E=l_{2}, x=(x_{1}, x_{2}, \ldots) 
    $

    $$ f(ax+by) = \sum_{k=1}^{\infty} \frac{(ax_{k} + by_{k})}{k} = 
    a \sum_{k=1}^{\infty} \frac{x_{k}}{k} + b \sum_{k=1}^{\infty} \frac{y_{k}}{k} = 
    af(x) + bf(y)$$
    
    $$f(x)= \sum_{k=1}^{\infty} \frac{x_{k}}{k} \le 
    \sqrt{\sum_{k=1}^{\infty}x_{k}^2} \sqrt{\sum_{k=1}^{\infty}\frac{1}{k^2}} - 
    \text{Нерівність Гельдера} $$

    $$f(x)\le \|x\| \frac{\pi^2}{6} \Rightarrow \|f\| = \frac{\pi^2}{6} $$

    $
    1.3) f(x)=\sum_{k=1}^{\infty} \frac{x_{k}}{k}, E=l_{1}, x=(x_{1}, x_{2}, \ldots) 
    $

    $$
    |f(x)|= \sum_{k=1}^{\infty} \frac{|x_{k}|}{k} \le  
    $$

    $
    \text { 1.4) } f(x)=\sum_{k=1}^{\infty} \frac{x_{k}}{k}, E=l_{4}, x=(x_{1}, x_{2}, \ldots) 
    $

    $$ |f(x)| = \sum_{k=1}^{\infty}\frac{x_k}{k} \le 
    \sqrt[4]{\sum_{k=1}^{\infty}x_k^4}
    (\sum_{k=1}^{\infty}k^{-\frac{4}{3}})^{\frac{3}{4}} $$
    $$\|f\|=(\sum_{k=1}^{\infty}k^{-\frac{4}{3}})^{\frac{3}{4}} $$

\end{tcolorbox}

\begin{tcolorbox}
    $
    \text { 1.5) } f(x)=\int_{0}^{1} t^{2} x(t) \mathrm{d} t, E=C([0,1]) 
    $

    $$ \text{Інтеграл - лінійна функція, отже і наш функціонал лінійний} $$
    $$ |f(x)| = \int_0^2 t^2 |x(t)| \le \max_{t \in[0,1]}|x|\int_{0}^{1}t^2dt = 
    \|x\| \frac{1}{3} \Rightarrow \|f\| = \frac{1}{3}$$

    $
    \text { 1.6) } f(x)=\int_{0}^{1} t^{2} x(t) \mathrm{d} t, E=L_{2}([0,1]) 
    $

    $$ |f(x)| = \int_0^1t^2x(t)dt \le 
    \sqrt{\int_0^1x^2(t)dt} \sqrt{\int_0^1t^4dt} = 
    \|x\|\frac{\sqrt{5}}{5} \Rightarrow \|f\|=\frac{\sqrt{5}}{5}$$

    $
    \text { 1.7) } f(x)=\int_{-1}^{1} \operatorname{sgn}(t) x(t) \mathrm{d} t, E=L_{2}([-1,1]) 
    $

    $$ |f(x)| = |\int_{-1}^1sgn(t)x(t)dt| \le 
    \sqrt{\int_{-1}^1x^2(t)dt} \sqrt{\int_{-1}^{1}sgn^2(t)dt} = 
    \|x\| 2 \Rightarrow \|f\| = 2 $$

    $
    \text { 1.8) } f(x)=\int_{-1}^{1} \operatorname{sgn}(t) x(t) \mathrm{d} t, E=C([-1,1]) 
    $
    
    $$ |f(x)| = |\int_{-1}^1sgn(t)x(t)dt| \le 
    \max_{t \in [-1, 1]} |x(t)| \int_{-1}^1sgn(t)dt ??? $$


    $
    \text { 1.9) } f(x)=\int_{-1}^{1} x^{2}(t) \mathrm{d} t, E=C([-1,1]) 
    $

    $$ |f(x)| =  $$

    $
    \text { 1.10) } f(x)=x(0), E=C([-1,1]) 
    $

    $$|f(x)| = |x(0)| \le \max_{t\in[-1, 1]}x(t) = \|x\| \Rightarrow \|f\| = 1$$

    $
    \text { 1.11) } f(x)=2 x(-1)-x(1), E=C([-1,1]) 
    $

    $$ |f(x)| = |2x(-1) - x(1)| \le 3\|x\| (\text{досягається при } x(-1) = -x(1) = max x) 
    \|f\| = 3 $$

    $
    \text { 1.12) } f(x)=x(0), E=C([0,1]),|x \|_{E}=\int_{0}^{1}| x(\xi) \mid d \xi 
    $

    $
    \text { 1.13) } f(x)=x^{\prime}(0)+x(1), E=C(1)([0,1]),|x|=\max _{t \in[0,1]}|x(t)|+\max _{t \in[0,1]}|x^{\prime}(t)|
    $

    $$|f(t)| = |x`(0) + x(1)| \le \max _{t \in[0,1]}|x(t)|+\max _{t \in[0,1]}|x^{\prime}(t)|=
    \|x\| \Rightarrow \|f\| = 1 $$


\end{tcolorbox}

\begin{tcolorbox}[title=Завдання 2]
    2. Нехай $f: E \rightarrow \mathbb{R}$ - ненульовий ЛНФ. Довести, що область значень $f$ співпадає з $\mathbb{R}$.
\end{tcolorbox}