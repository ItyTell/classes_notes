\section*{Розв'язання}


\begin{tcolorbox}[title=Завдлання 1]
    Дослідити послідовності на слабку збіжність.

    $
    1.1) x^{(n)}=(\underbrace{0,0, \ldots, 0}_{n-1}, 1, \frac{1}{2},
    \frac{1}{3}, \frac{1}{4}, \ldots), E=l_{p}, 1<p<\infty $


    $$\forall f \in E^*, \exists a \in l_q: f(x) = 
    \sum_{k=1}^\infty a^k x^k$$
    $$\text{Тоді } f(x_n) = \sum_{k = n}^{\infty} \frac{a_k^q}{(k - n + 1)^q},
    \sqrt[q]{\sum_{k=1}^{\infty} a_k^q} < \infty \Rightarrow a_k \rightarrow 0, 
    q > 1$$

    $$ \sum_{k=n-1}^{\infty}  \frac{a_k^q}{(k - n + 1)^q} < 
    \sum_{k=n-1}^{\infty}  \frac{1}{(k - n + 1)^d}, d>1 \Rightarrow f(x_n) \rightarrow
    f(0) = 0 $$


    $
    1.2) x^{(n)}=(\underbrace{0,0, \ldots, 0}_{n-1}, \frac{1}{n}, 
    \frac{1}{n+1}, \frac{1}{n+2}, \frac{1}{n+3}, \ldots), E=l_{p}, 1<p<\infty
    $

    $$\forall f \in E^* |f(x_n)| < |f(x_n^*)| 
    \Rightarrow f(x_n) \rightarrow 0$$
    Де $x_n^*$ з минулого номера 

    $
    1.3) x^{(n)}=t^{n}, E=L_{p}([0,1]), 1 \leq p<\infty
    $

    $$ \forall f \in L_p^* \exists a \in L_q: f(x)=\int_{0}^{1} x(t)a(t)dt $$
    $$ 0 < \int_{0}^{1} x_n(t)a(t)dt \le \max_{t \in [0, 1]} a \int_{0}^{1} x_n(t)dt $$
    Звідси $f(x_n) \rightarrow 0, x_n \rightarrow 0$


    $
    1.4) x^{(n)}=\sqrt{n} \chi_{[0, \frac{1}{n}]}, E=L_{p}([0,1]), 1 \leq p \leq \infty
    $

    Відкинемо значення $p$ при яких $\|x_n\|$ необмежена 

    $$ \|x_n\| = \sqrt[p]{\int_{0}^{1}x_n^p(t) dt} = 
    \sqrt[p]{\int_{0}^{\frac{1}{n}} \sqrt{n^p}dt } =
    n^{\frac{p - 1}{2p}} $$

    При $p>1, \|x_n\| \rightarrow \infty$


    Залишається перевірити $p = 1$
    $$\int_{0}^{\xi} x_n(t)dt \le \frac{1}{\sqrt{n}} \rightarrow 0$$

\end{tcolorbox}

\begin{tcolorbox}[title=Завдання 2]

    Дослідити послідовності функціоналів на *-слабку збіжність та перевірити чи збіжна ця послідовність за нормою.

    $
    \text { 2.1) } f_{n}(x)=\int_{0}^{1} x(t) \cos (2 \pi n t) d t, E=L_{2}([0,1])
    $

    $$f_n = \int_{0}^{2\pi} x(\frac{t}{2\pi}) \cos (n t) d t$$

    Якщо подивитись на графік косинуса стає очевидно, що зі з збільшенням $n$ густина 
    хвильок стає більшою, при ліміті $n \rightarrow \infty$ в інтегралі кожне значення 
    $x$ на відрізку $[0, 1]$ нівелюється протилежним (з множником відповідної хвилі) 
    і залишиться лише множина міри нуль, тобто $f_n(x) \rightarrow 0$.

    Це якесь не дуже строге доведення, за нормою взагалі хз як доводити.


    $
    \text { 2.2) } f_{n}(x)=n \int_{0}^{\frac{1}{2 n}}(1-2 n t) x(t) d t, E=C([0,1])
    $
    

    З геометричного визначення інтегралу маємо:

    $$f_n(x) = n \int_{0}^{\frac{1}{2n}}(1-2nt)x(t)dt \rightarrow 
    n frac{1}{2n}x(0) = \frac{x(0)}{2} $$

    
    $
    \text { 2.3) } f_{n}(x)=n \int_{0}^{1} t^{n} x(t) d t, E=C([0,1])
    $

    $$f_n(x) = n \int_{0}^{1} t^{n} x(t) d t \rightarrow 0 $$
\end{tcolorbox}