\section*{Розв'язання}


\begin{tcolorbox}[title=Завдлання 1]
    1. Нехай $G=\{x \in C([0,1]): x(0)=0\}$. 
    Побудувати лінійний неперервний функціонал на $C([0,1])$ який дорівнює нулю на $G$ 
    і набуває значення 2 на функціi $x_0(t)=t+1, t \in[0,1]$.
    $$f(x) = c x(0), \text{ } f(x_0) = c = 2 \Rightarrow f(x) = 2 x(0)$$
    Очевидно данний функціонал лінійний та неперервний.
\end{tcolorbox}

\begin{tcolorbox}[title=Завдлання 2]
    Довести, що існує ненульовий функціонал $F \in(L_{\infty}([-1,1]))^*$ такий, що
    $$
    F(x)=x(0), \text { при } x \in C([-1,1]) .
    $$

    Можливо за теоремою Ханна-Банаха можна продовжити функціонал з $C([-1,1])$ на $L_{\infty}([-1,1])$, а отже він існує.
\end{tcolorbox}

\begin{tcolorbox}[title=Завдлання 3]
    Чи правильно, що у ЛНП $X$ елементи $x$ та $y$ є рівними, 
    якщо рівність $f(x)=f(y)$ має місце для всіх $f \in X^*$ ?
    

    Нехай $x \neq y, z = x-y, \forall f \in X^* f(x) = f(y) - f(z)$


    Твердження з задачі буде виконуватись лише при $f(z) \neq 0$ де $z \neq 0$

    припустимо, що $\exists z\neq 0 \in X, f(z) = 0 $


    введемо функціонал $f_0(\alpha z) = \alpha, \forall x \in X f_0(x) = 0$, якщо 
    $f(x) \neq 0$
    з неперервності лінійного відображення $f_0$ маємо що воно належить $X^*$ тобто маємо 
    протиріччя.


    Тож твердження з умови правильне
\end{tcolorbox}

\begin{tcolorbox}[title=Завдлання 4]
    Чи правильно, що рефлексивний ЛНП є банаховим?
    

    Так за визначенням рефлексивного простору, він є банаховим + при канонічному вкладенні
    збігається з **спряженим.
\end{tcolorbox}