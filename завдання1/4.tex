
\chapter{Завдання \theHchapter}


\begin{tcolorbox}[title=Завдання]
    
    Нехай C — непорожня замкнена опукла пiдмножина гiльбертового 
    простору $ H, x \in H $ \textbackslash $C $.
    
    
    Доведiть, що $ \exists p \in H$ \textbackslash $\{0\} $ такий, що
    $$ \s (p, y) < (p, x) $$

    
\end{tcolorbox}

\center{\bfseries Розв'язання:}


\center{\bfseries Повне доведення:}

$ C $ замкнена множина отже замість $\sup$ можна інтерпритувати умову як

$\forall y \in C, \exists p \in H \backslash \{0\} \quad 
(p, y) < (p,x) $


Доведення проводимо від супротивного. Нехай $\exists y \in C$ такий що
$\forall p \in H \backslash \{0\} \quad (p, y) \geq (p, x)$


$ (p, y) - (p, x) = (p, y - x) \geq 0 $


$ (-p, y - x) \le 0 $


Подивившись на теорему 2 другої лекції (а саме на її другий пункт)
стає очевидно, що за $-p$ варто взяти $P_Cx - x$.
Тоді отримаємо 


$ (P_Cx - x, y - x) = (P_Cx - x, y - P_Cx + P_Cx - x) = $


$=(P_Cx - x, y - P_Cx) +(P_Cx - x, P_Cx - x) \le 0 $


За теоремою 2 з лекції 2 (пункт 2) $(P_Cx - x, y - P_Cx) \geq 0 $ 


$(P_Cx - x, P_Cx - x) = \|P_Cx - x\|^2 > 0$ адже $x \in H \backslash C,
x \neq P_Cx$


Доведено 




\center{\bfseries Швидке доведення:}



Покладемо $p = x - P_C x \neq 0$. Для довiльного $y \in C$ 
з теореми 2 другої лекції (пункт 2) маємо $ \forall y \in C $


$0 \geq (x - P_C x, y - P_C x)=(p, y - x + p)=(p, y - x) + \|p\| ^2$.


$(p, y) - (p, x) + \|p\| ^2 \le 0, \|p\| ^ 2 \geq 0 \Rightarrow 
(p, y) \le (p, x), \|p\| ^ 2 > 0 $ 


Отже $ \s (p, y) < (p, x) $,
що i треба було довести.

