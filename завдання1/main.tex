\documentclass[a4paper]{extreport}
\usepackage[T2A]{fontenc}
\usepackage[utf8]{inputenc}
\usepackage[ukrainian]{babel}
\usepackage{caption, hyperref, tempora, authblk, setspace, amsmath, setspace,
            titlesec, geometry, indentfirst, tempora, tikz, pgfplots,
            datetime, cancel, tcolorbox}



\pgfplotsset{width=10cm,compat=1.9}

\onehalfspacing



\titleformat{\chapter}{\fontsize{18}{14}\bfseries\center\MakeUppercase}{}{0.5em}{}
\titleformat{\section}{\fontsize{14}{14}\bfseries}{\thesection}{0.5em}{}
    
\geometry{left = 3cm, right = 1.5cm, top = 2cm, bottom = 2cm}
\renewcommand {\baselinestretch} {1.25}

\setlength{\parindent}{3em}
\usetikzlibrary{positioning}

\hypersetup{
    colorlinks=true,
    linkcolor=orange,
    filecolor=magenta,      
    urlcolor=red,
    pdftitle={Math-Phusics},
}

\urlstyle{same}


\title{Завдання 1 з премету Спецкурс для ОМ-3}
\author{Коломієць Микола}
\date{\today}
\renewcommand{\familydefault}{phv}

\tcbset{
        colback=red!5!white,colframe=red!50!black,
        fontlower=\spacing{0}\linespread{3000.8}\sffamily\bfserie,
        code={\doublespacing}
        }




\def\m{\displaystyle\min_{z \in C}}

\def\k{\displaystyle<^?}

\begin{document}

    \maketitle

    \tableofcontents

    \pagebreak

    \chapter{Завдання \theHchapter}

    \begin{tcolorbox}[title=Завдання]
        Нехай C — непорожня замкнена опукла пiдмножина гiльбертового простору H. Доведiть, що
        $$ \| P_Cx - P_Cy  \|^ 2 \le
        \| x - y \| ^ 2 - 
        \| (x - P_Cx) - (y - P_Cy) \| ^ 2, \ \forall x, y \in H $$
    \end{tcolorbox}

    \center{\bfseries Розв'язання:}


    $ \| P_Cx - P_Cy  \|^ 2 = \| P_Cx - x - P_Cy + y + x - y  \|^ 2 = 
    \| x - y - ((x - P_Cx) - (y - P_Cy))  \|^ 2 = $


    $= \| x - y \| ^ 2 - 2 ( x - y,(x - P_Cx) - (y - P_Cy) ) + 
    \| (x - P_Cx) - (y - P_Cy) \| ^ 2 $


    Порівняємо з правою частиною нерівності:
    
    
    $ \cancel{\| x - y \| ^ 2} - 2 ( x - y,(x - P_Cx) - (y - P_Cy) ) + 
    \| (x - P_Cx) - (y - P_Cy) \| ^ 2 \stackrel{?}{\le} 
    \cancel{\|x - y\|^2} - \| (x - P_Cx) - (y - P_Cy) \| ^ 2 $



    $\|(x - P_Cx) - (y - P_Cy) \| ^ 2 \stackrel{?}{\le} 
    (x - y,(x - P_Cx) - (y - P_Cy))$


    $ \|(x - P_Cx) - (y - P_Cy) \| ^ 2 = 
    ((x - P_Cx) - (y - P_Cy), (x - P_Cx) - (y - P_Cy)) =$
    

    $= (x - y, (x - P_Cx) - (y - P_Cy)) - (P_Cx - P_Cy, (x - P_Cx) - (y - P_Cy)) $


    $ \cancel{(x - y, (x - P_Cx) - (y - P_Cy))} - (P_Cx - P_Cy, (x - P_Cx) - (y - P_Cy)) 
    \stackrel{?}{\le} \cancel{(x - y,(x - P_Cx) - (y - P_Cy))} $


    $(P_Cx - P_Cy, (x - P_Cx) - (y - P_Cy)) \stackrel{?}{\geq} 0 $


    $(P_Cx - P_Cy, (x - P_Cx) - (y - P_Cy)) = 
    (P_Cx - P_Cy, x - y) - (P_Cx - P_Cy,  P_Cx - P_Cy) = $


    $=(P_Cx - P_Cy, x - y) - \|P_Cx - P_Cy \|^2 \stackrel{?}{\geq} 0 $


    Якщо застосувати другий пункт теореми 2 з лекції 2 
    при $1)z = P_Cx, x = x, y = P_Cy, 2)z = P_Cy, x = y, y = P_Cx $:


    Отримаємо:


    $ (P_Cx - x, P_Cy - P_Cx) \geq 0 , (P_Cy - y, P_Cx - P_C) \geq 0$


    І якщо складемо їх отримаємо нашу нерівеість:


    $(P_Cx - P_Cy, x - y) \geq \|P_Cx - P_Cy \|^2  $, що і завершує доведення.



    \chapter{Завдання \theHchapter}


    \begin{tcolorbox}[title=Завдання]
        Нехай $\{e_n\}_{n \in \mathbf{N}} $ — злiченна ортонормована система елементiв 
        гiльбертового простору H та
        
        
        C = з.л.о. $\{e_n\}$. Доведiть, що
        $$ P_Cx = \sum_{n = 1}^{\infty}(x, e_n)e_n, x \in H $$
    \end{tcolorbox}

    \center{\bfseries Розв'язання:}



\end{document}