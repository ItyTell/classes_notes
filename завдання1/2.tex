
\chapter{Завдання \theHchapter}


\begin{tcolorbox}[title=Завдання]
    Нехай $\{e_n\}_{n \in \mathbf{N}} $ — злiченна ортонормована система елементiв 
    гiльбертового простору H та
    
    
    C = з.л.о. $\{e_n\}$. Доведiть, що
    $$ P_Cx = \sum_{n = 1}^{\infty}(x, e_n)e_n, x \in H $$
\end{tcolorbox}

\center{\bfseries Розв'язання:}

Нехай $ x \in H $. Тоді розклад вектора $P_Cx$ можна записати так


$ P_Cx = \sum_{n = 1}^{\infty}(P_Cx, e_n)e_n = 
\sum_{n = 1}^{\infty}(P_Cx - x + x, e_n)e_n =$


$=\sum_{n = 1}^{\infty}((P_Cx - x, e_n) + (x, e_n) )e_n =
\sum_{n = 1}^{\infty}(P_Cx - x, e_n)e_n +\sum_{n = 1}^{\infty}(x, e_n)e_n$


C - з.л.о., тобто він і замкнений і лінійний і очевидно $ e_n \in C $.


Тоді за теоремою 4 другої лекції $ (P_Cx - x, e_n) = 0 $


$ P_Cx = \sum_{n = 1}^{\infty}\underbrace{(P_Cx - x, e_n)}_{\mbox{0}}e_n 
+ \sum_{n = 1}^{\infty}(x, e_n)e_n = 
\sum_{n = 1}^{\infty}(x, e_n)e_n $ 


Доведено
