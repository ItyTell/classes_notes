\section*{Вступ}

Нещасні випадки на виробництві та профзахворювання залишаються серйозною проблемою в Україні. Щорічно реєструється близько 10 тисяч нещасних випадків на виробництві, з них близько 200 зі смертельним наслідком. Крім того, щорічно близько 7 тисяч людей визнаються профхворими.

У зв'язку з цим, важливо дослідити права та обов'язки роботодавця як страхувальника від нещасних випадків у законодавстві України. Це дослідження допоможе роботодавцям краще зрозуміти свою роль у системі загальнообов'язкового державного соціального страхування від нещасного випадку на виробництві та профзахворювань, а також сприятиме зменшенню кількості нещасних випадків на виробництві та профзахворювань.

Метою даного реферату є дослідження прав та обов'язків роботодавця як страхувальника від нещасних випадків у законодавстві України.

В рефераті буде досліджено такі питання:
\begin{enumerate}

    \item Поняття "роботодавець як страхувальник" та його правовий статус.

    \item Права та обов'язки роботодавця як страхувальника.

    \item Відповідальність роботодавця за порушення законодавства про загальнообов'язкове державне соціальне страхування від нещасного випадку на виробництві та профзахворювань.

    \item Практика застосування законодавства про загальнообов'язкове державне соціальне страхування від нещасного випадку на виробництві та профзахворювань.

\end{enumerate}
Результати дослідження будуть узагальнені у висновках.



\section*{Основна частина}
Стаття 8. Права, обов’язки та відповідальність роботодавця як страхувальника

1. Роботодавець як страхувальник має право на:

1) безоплатне отримання в уповноваженому органі управління інформації про порядок використання страхових коштів;

2) отримання інформації про результати проведення перевірки використання страхових коштів;

3) судовий захист своїх прав.

2. Роботодавець зобов’язаний:

1) здійснювати застрахованим особам у разі настання страхового випадку відповідний вид страхових виплат та надання соціальних послуг згідно з цим Законом;

2) вести облік страхових коштів і своєчасно надавати страховику встановлену звітність щодо цих коштів;

3) під час перевірки правильності використання страхових коштів та достовірності поданих роботодавцем даних надавати посадовим особам уповноваженого органу управління необхідні документи та пояснення з питань, що виникають під час перевірки;

4) подавати у встановленому порядку відповідно до законодавства відомості про:

розмір заробітної плати працівників та використання ними робочого часу;

використання страхових коштів за іншими визначеними цим Законом напрямами в порядку, встановленому Кабінетом Міністрів України;

5) інформувати про кожний нещасний випадок або професійне захворювання на підприємстві, в установі, організації;

6) допускати посадових осіб уповноваженого органу управління для здійснення перевірок правильності використання страхових коштів, контролю за веденням і достовірністю обліку та звітності щодо їх надходження та використання за наявності направлення та/або наказу про перевірку та посвідчення осіб;

7) інформувати працівників підприємства, установи, організації про засади соціального страхування, підстави та порядок здійснення страхових виплат, а також про державні фінансові гарантії медичного обслуговування населення та реабілітаційну допомогу;

8) подавати звітність до уповноваженого органу управління у строки, в порядку та за формою, що встановлені Кабінетом Міністрів України;

9) повернути уповноваженому органу управління суму здійснених страхових виплат та вартість наданих соціальних послуг потерпілому на виробництві у разі невиконання своїх зобов’язань щодо сплати страхових внесків.

3. Достовірність зазначених у документах відомостей перевіряється уповноваженим органом управління. У разі подання недостовірних відомостей, використання роботодавцем страхових коштів з порушенням встановленого порядку роботодавець добровільно чи на підставі рішення суду повинен відшкодувати страховику заподіяну шкоду.

4. Роботодавцеві забороняється вчиняти будь-які дії, що можуть призвести до прийняття ним разом із застрахованою особою спільного рішення, яке може в подальшому зашкодити цій особі або членам її сім’ї реалізувати своє право на страхові виплати та соціальні послуги відповідно до цього Закону.

5. Роботодавець несе відповідальність за:

1) порушення порядку використання страхових коштів, несвоєчасне або неповне їх повернення;

2) несвоєчасне подання або неподання відомостей, встановлених цим Законом;

3) подання недостовірних відомостей про використання страхових коштів;

4) шкоду, заподіяну застрахованим особам або уповноваженому органу управління внаслідок невиконання або неналежного виконання обов’язків, визначених цим Законом.

6. У разі порушення порядку використання страхових коштів роботодавець відшкодовує уповноваженому органу управління в повному обсязі неправомірно витрачену суму страхових коштів та/або вартість наданих соціальних послуг і сплачує штраф у розмірі 50 відсотків такої суми.

За несвоєчасне повернення або повернення не в повному обсязі страхових коштів на страхувальників та інших отримувачів коштів соціального страхування у зв’язку з тимчасовою втратою працездатності та від нещасного випадку накладається штраф у розмірі 10 відсотків несвоєчасно повернутих або повернутих не в повному обсязі страхових коштів.

Одночасно на суми несвоєчасно повернутих або повернутих не в повному обсязі страхових коштів і штрафних санкцій нараховується пеня в розмірі 0,1 відсотка зазначених сум коштів, розрахована за кожний день прострочення платежу.

7. Своєчасно не сплачені фінансові санкції та адміністративні штрафи стягуються до бюджету уповноваженого органу управління в порядку, встановленому законом.

8. Право застосовувати фінансові санкції та накладати адміністративні штрафи від імені уповноваженого органу управління мають керівники територіальних органів уповноваженого органу управління та їх заступники.