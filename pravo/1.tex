\section*{Вступ}

Нещасні випадки на виробництві та профзахворювання залишаються серйозною проблемою в Україні. Щорічно реєструється близько тисячі нещасних випадків на виробництві, з них близько 400 смертельні.

У зв'язку з цим, важливо дослідити права та обов'язки роботодавця як страхувальника від нещасних випадків у законодавстві України. 
Цей реферат допоможе роботодавцям краще зрозуміти свою роль у системі страхування від нещасного випадку.

Метою даного реферату є дослідження прав та обов'язків роботодавця як страхувальника від нещасних випадків у законодавстві України.

В рефераті буде досліджено такі питання:
\begin{enumerate}

    \item Поняття "роботодавець як страхувальник" та його правовий статус.

    \item Відповідальність роботодавця за порушення законодавства про загальнообов'язкове державне соціальне страхування від нещасного випадку.

\end{enumerate}
Результати дослідження будуть узагальнені у висновках.



\section*{Основна частина}

\subsection*{Права}

Роботодавець має право на безоплатне отримання інформації про порядок використання страхових коштів та про результати перевірки їх використання. Тако він має право на судовий захист своїх прав.

\subsection*{Обов'язки}

Роботодавець зобов'язаний здійснювати стахові виплати застрахованим особам у разі страхового випадку, веси облік страхових коштів і своєчасно надавати страховику встановлену звітність щодо цих коштів, інформувати про кожний нещасний випадок або професійне захворювання на підприємстві, в установі, організації, допускати посадових осіб уповноваженого органу управління для здійснення перевірок правильності використання страхових коштів, контролю за веденням і достовірністю обліку та звітності щодо їх надходження та використання за наявності направлення та/або наказу про перевірку та посвідчення осіб.


Також роботодавець повинен інформувати працівників підприємства, установи, організації про засади соціального страхування, підстави та порядок здійснення страхових виплат, а також про державні фінансові гарантії медичного обслуговування населення та реабілітаційну допомогу - тобто працівники мають знати ці деталі стосовно власного страхування.


І роботодавець має повернути уповноваженому органу управління суму здійснених страхових виплат та вартість наданих соціальних послуг потерпілому на виробництві у разі невиконання своїх зобов'язань щодо сплати страхових внесків.


\subsection*{Відповідальність}


У разі подання недостовірних відомостей, використання роботодавцем страхових коштів з порушенням встановленого порядку роботодавець добровільно чи на підставі рішення суду повинен відшкодувати страховику заподіяну шкоду.


Також роботодавцю заборонено змовлятись з застрахованими працівниками з метою отримати стахових виплат внаслідок "нещасного випадку".

Роботодавець несе відповідальність за будь-яке порушення порядку використання коштів, несвоєчасно подання відомістей стосовно них та подання недостовірних відомостей про використання страхових коштів. Також він несе відповідальність за шкоду, заподіяну застрахованим особам або уповноваженому органу управління внаслідок невиконання або неналежного виконання обов'язків, визначених цим Законом.

У разi порушення порядку використання страхових коштiв роботодавець вiдшкодовує уповно-
важеному органу управлiння в повному обсязi неправомiрно витрачену суму страхових коштiв та/або
вартiсть наданих соцiальних послуг i сплачує штраф у розмiрi 50 вiдсоткiв такої суми.

За несвоєчасне повернення або повернення не в повному обсязi страхових коштiв на страхувальникiв та iнших отримувачiв коштiв соцiального страхування у зв’язку з тимчасовою втратою працездатностi та вiд нещасного випадку накладається штраф у розмiрi 10 вiдсоткiв несвоєчасно повернутих або повернутих не в повному обсязi страхових коштiв

Своєчасно не сплачені фінансові санкції та адміністративні штрафи стягуються до бюджету уповноваженого органу управління в порядку, встановленому законом.


\section*{Висновки}

Роботодавець має право на безоплатне отримання інформації про порядок використання страхових коштів та про результати перевірки їх використання. Тако він має право на судовий захист своїх прав.


Роботодавець зобов'язаний здійснювати стахові виплати застрахованим особам у разі страхового випадку, веси облік страхових коштів і своєчасно надавати страховику встановлену звітність щодо цих коштів, інформувати про кожний нещасний випадок або професійне захворювання на підприємстві, в установі, організації, допускати посадових осіб уповноваженого органу управління для здійснення перевірок правильності використання страхових коштів, контролю за веденням і достовірністю обліку та звітності щодо їх надходження та використання за наявності направлення та/або наказу про перевірку та посвідчення осіб.


У разі порушення більшості обов'язків (запізнення з ними) роботодавець має відшкодувати страховику заподіяну шкоду та сплатити певний штраф, це залежить від порушення.

У роботодавцяф велика кількість обов'язків та відповідальності, тому можна зробити висновок, що працівники в Україні є досить захищеними від нещасних випадків на виробництві в плані стахових виплат.

\section*{Нормативно-правова база}

1. Про загальнообов’язкове державне соціальне страхування: Закон України від 02.09.99.


Відомості Верховної Ради України. – 1999. – No
46-47. – Ст. 403. стаття 8


URL: \url{https://zakon.rada.gov.ua/laws/show/1105-14}