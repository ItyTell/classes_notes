\chapter{Застосування до розв’язання модельної задачi}

Автор розділу - Коломієць Микола

\section{Постановка задачi}

\begin{equation}
    \begin{cases}
        - (\frac{\partial^2 u}{\partial x_1 ^ 2} 
        + \frac{\partial ^2 u}{\partial x_2^2}) = \varphi(x,y) \\
        0 < x_1 < 1, 0 < x_2 < 1 \\
        u(0,y) = 0 \\
        u(1,y) = A e^{By}\sin \omega \\
        u(x,0) = A \sin \omega x \\
        u(x,1) = A e^B \sin \omega x \\
        \varphi(x,y) = A e^{By} (\omega^2 - B^2)\sin \omega x
    \end{cases}
\end{equation}


Розв'язок: $u(x, y) = A e^{By}\sin(\omega x)$


Початкові значення: $A=B=1, \omega = \pi$

\section{Зведення}

Візьмемо прямокутну сітку:
$$ \omega = \{ x_{i,j} = (ih_1, jh_2) \in G, 
0 \le i \le M, 0 \le j \le N, h_1 = 1 / M, h_2 = 1 / N \} $$
з границею $\gamma$, на прямокутнику $G = \{0 \le x, y \le 1 \}$
треба вирішити різнісну задачу Діріхле для рівняння Пуассона: 
$$ y_{x_1, x_1} + y_{x_2, x_2} = -\varphi(x), x\in \omega $$
$$ y(x) = g(x), x \in \gamma $$


Після зведення рівняння до рівняння першого порядку та подальшою 
апроксимацією цієї системи різницевою схемою, отримаємо задачу з минулого 
розділу:

$$ -Y_{j-1} + C Y_j - Y_{j+1} = F_j $$
$$ Y_0 = F_0, Y_N = F_N $$

Тут $Y_j$-вектор розмірності $M-1$ елементами якого є значення сіткової функції 
$y(i, j)= y(x_{i,j})$ в внутрішніх $j$-тих вузлах сітки $\omega$:

$$ Y_j = (y(1, j), y(2, j), \dots , y(M-1, j)), 0\le j \le N $$

$C$- квадратна матриця розмірності $M-1$ на $M-1$, 
яка відповідає різницевому оператору $\Delta$:

$$\Lambda y =2 y-h_2^2 y_{\bar{x}_1 x_1}, \quad h_1 \le x_1 \le l_1-h_1,$$ 
$$y =0, \quad x_1=0, l_1$$

Отже матриця $C$- є трьохдоганальною симетричною матрицею і тоді:


$$CY_j=(\Lambda y(1, j), \quad \Lambda y(2, j), \ldots, \Lambda y(M-1, j))$$


$F_j$- права частина, вектор розмірності $M-1$: 



1) для $j = 1, 2, \dots, N-1$
$$F_j=\left(h_2^2 \bar{\varphi}(1, j), h_2^2 \varphi(2, j), \ldots, h_2^2 \varphi(M-2, j), h_2^2 \bar{\varphi}(M-1, j)\right),$$


Де 
$$\begin{aligned} \bar{\varphi}(1, j) & =\varphi(1, j)+\frac{1}{h_1^2} g(0, j), \\ \bar{\varphi}(M-1, j) & =\varphi(M-1, j)+\frac{1}{h_1^2} g(M, j) .\end{aligned}$$


2) для $j = 0, N$:
$$F_j=(g(1, j), g(2, j), \ldots, g(M-1, j)).$$

Далі застосовуємо наш алгоритм і отриуємо розвязок
