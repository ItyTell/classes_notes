\chapter{Методи матричної прогонки}

Автор розділу - Депенчук Марічка

\section{Загальний огляд}

Метод матричної прогонки, також відомий як алгоритм Томаса, $\epsilon$ ефективним чисельним інструментом 
для розв'язання систем лінійних рівнянь, зокрема тих, що виникають при дискретизації диференціальних рівнянь. 
Цей метод використовується для систем з тридіагональною матрицею коефіцієнтів, що є типовими у багатьох застосуваннях, 
наприклад, у фізиці та інженерії.

\section{Теорія}

Тридіагональна матриця виглядає наступним чином, де $a_{i}, b_{i}$, i $c_{i}$ діагональні елементи, $a d_{i}$ - елементи вектора правих частин:

$$
\left[\begin{array}{ccccc}
b_{1} & c_{1} & & & 0 \\
a_{2} & b_{2} & c_{2} & & \\
& a_{3} & b_{3} & \ddots & \\
& & \ddots & \ddots & c_{n-1} \\
0 & & & a_{n} & b_{n}
\end{array}\right]\left[\begin{array}{c}
x_{1} \\
x_{2} \\
x_{3} \\
\vdots \\
x_{n}
\end{array}\right]=\left[\begin{array}{c}
d_{1} \\
d_{2} \\
d_{3} \\
\vdots \\
d_{n}
\end{array}\right]
$$

Метод прогонки заснований на двоетапному процесі: прямому ході, де проводиться елімінація знизу вгору, та зворотньому ході, де здійснюється підстановка від останнього рівняння до першого.

Прямий хід:
- змінені коефіцієнти Pi та Qi визначаються для оптимізації розрахунків. Ці коефіцієнти використовуються для тимчасового зберігання проміжних значень, що спрощує зворотний хід.
- Розрахунок коефіцієнтів:

$P_{i}=\frac{-c_{j}}{b_{i}+a_{i} P_{i-1}}, \quad Q_{i}=\frac{d_{i}-a_{i} Q_{i-1}}{b_{i}+a_{i} P_{i-1}}$
- 3 цього видно, що Рi - модифікований вплив попереднього рядка на наступний, а Qi - модифікована права сторона з урахуванням уже врахованих змін.

Зворотний хід:
\begin{itemize}
    \item Використовуючи раніше обчислені коефіцієнти Pi та Qi, зворотний хід забезпечує швидке визначення всіх хі.
    \item Формула для зворотнього ходу:
    $x_{n}=Q_{n}, \quad x_{i}=Q_{i}+P_{i} x_{i+1}$
    
    
    Ця формула дозволяє обчислити кожне $\mathrm{x}_{\mathrm{i}}$ починаючи з кінця системи $\mathrm{i}$ просуваючись до її початку.
\end{itemize}


\section{Теорія розширеного методу матричної прогонки}

Розглянемо узагальнений метод прогонки для системи лінійних рівнянь $з$ блочно-тридіагональною матрицею, яка може виникнути, наприклад, при розв'язуванні багатовимірних диференційних рівнянь у частинних похідних.


Система може бути представлена у формі:
$A_{j} Y_{j-1}+C_{j} Y_{j}+B_{j} Y_{j+1}=F_{j}$,
де кожен з Aj, Bj, i Cj є блоками матриці коефіцієнтів, а Yj і Fј відповідно блоки невідомих змінних та вільних членів системи.

\vspace{1cm}

Прямий хід:
\begin{itemize}
    \item Ініціалізуємо коефіцієнти $\alpha_1$ та $\beta_1$, які відповідають початковим блокам системи:


    $\alpha_{1}=C_{0}^{-1} B_{0} \quad \beta_{1}=C_{0}^{-1} F_{0}$


    \item Далі, використовуючи рекурсивні формули, обчислюємо проміжні коефіцієнти для $\mathrm{j}=1,2, \ldots, \mathrm{N}-1$ :


    $\alpha_{j+1}=\left(C_{j}-A_{j} a_{j}\right)^{-1} B_{j}, \quad \beta_{j+1}=\left(C_{j}-A_{j} a_{j}\right)^{-1}\left(F_{j}+A, B_{j}\right)$
\end{itemize}

\vspace{1cm}

Зворотний хід:
\begin{itemize}
    \item Розпочинаємо з останнього блоку, визначаючи $Y_{N}$ :
    $$
    Y_{N}=\beta_{N+1}
    $$
    \item Використовуючи значення $Y_{N}$, обчислюємо $Y_{j}$ для $\mathrm{j}=\mathrm{N}-1, \mathrm{~N}-2, \ldots, 0$ з рекурсивної формули:

    $$
    Y_{j}=\alpha_{j+1} Y_{j+1}+\beta_{j+1}
    $$

\end{itemize}

\section{Висновок}
Метод матричної прогонки, який враховує блочну структуру матриці, дозволяє ефективно розв'язувати більш широкий клас систем лінійних рівнянь, які часто зустрічаються в багатовимірних чисельних задачах. Наведений алгоритм узагальнює класичний метод прогонки, зберігаючи його переваги щодо обчислювальної ефективності та простоти імплементації, проте водночас забезпечує можливість обробки систем з більш складною структурою. Основним обмеженням залишається вимога до тридіагональної структури матриці та до їі діагональної домінантності, без яких збільшується ризик чисельної нестабільності методу.