\chapter{Методи редукцiї (декомпозицiї)}

Автори розділу - Коломієць Микола, Депенчук Марічка

\section{Загальний огляд}

Метод редукції (декомпозиції) є одним з прямих методів для 
вирішення спеціальних векторних рівняннь до яких зводяться 
різницеві схеми для найпростіших еліптичних рівняннь. Данний метод має 
затрати на обчислення $O(MN\log_2(N))$ арифметичних дій.(Де M - розмірність 
вектора невідомих, а N кількість рівняннь)

\section{Задача}

Рівняння, на прикладі якого буде розглядатися данний метод:

$$-Y_{j-1} +CY_j - Y_{j+1} =F_j, \quad 1\le j \le N-1$$
$$Y_0 = F_0, \quad Y_N = F_N$$

Де $Y_j$-відповідний вектор невідомих, $F_j$ - задана права частина, 
C - задана права частина.(задача з розділу 4 також водиться до цього рівняння)

\section{Ідея алгоритму}

Ідея метода редукції полягає в послідовному виключені з рівняння невідомих 
$Y_j$ спочатку з непарними номерами $j$, потім з тих рівнянь, 
що лишились з номерами $j$, кратними 2 , потім 4 и так далі. За кожен крок 
процеса  Зменшується кількість невідомих, і якщо $N$ є стеменню 2 , тобто 
$N=2^n$, то в результаті процеса виключення залишиться одне рівняння, з 
котрого можна знайти $Y_{N / 2}$. 
Обернений хід метода полягає в послідовному находжені невідомих $Y_j$ 
спочатку з номерами $j$, кратними $N / 4$, потім $N / 8, N / 16$ і так далі.


Очевидно, що метод редукції це модифікований метод Гаусса, де виключення 
невідомих відбувається в спеціальному порядку. Вибраний порядок гарантує 
економію на обчисленнях при зменшені непотрібних шляхом запам'ятовування 
проміжнихрезультатів.

\section{Прямий хід}

\begin{enumerate}

    \item Задаються значення для 
    $q_j^{(0)}: q_j^{(0)}=F_j, j=1,2, \ldots, N-1$.

    \item Перший крок для $k=1$ робиться окремо по формулам, що враховють 
    початкові дані $p_i^{(0)} \equiv 0$. 
    
    
    Вирішуються рівняння для $p_j^{(1)}$ і обчислюються $q_j^{(1)}$ :
    $$
    \begin{aligned}
    & C p_j^{(1)}=q_j^{(0)}, \\
    & q_j^{(1)}=2 p_j^{(1)}+q_{j-1}^{(0)}+q_{j+1}^{(0)}, \quad j=2,4,6, \ldots, N-2 .
    \end{aligned}
    $$

    \item Для кожного фіксованого $k=2,3, \ldots, n-1$ обчислюється 
    та зберігається вектори:


    $$
    v_i^{(0)}=q_j^{(k-1)}+p_{j-2^{k-1}}^{(k-1)}+p_{j+2^{k-1}}^{(k-1)}, \quad j=2^k, 2 \cdot 2^k, 3 \cdot 2^k, \ldots, N-2^k .
    $$

    Потім при фіксованому $l=1,2,3, \ldots, 2^{k-1}$ для кожного 
    $j=2^k, 2 \cdot 2^k, 3 \cdot 2^k, \ldots, N-2^k$ вирішуються рівняння
    $$
    C_{l, k-1} v_j^{(l)}=v_j^{(l-1)}
    $$
    З однією незмінною матрицею , але різними правими частинами. В результаті будуть знайдені вектори $v_j^{\left(2^{k-1}\right)}$. 
    Вектори $p_j^{(k)}$ и $q_j^{(k)}$ обчислюються по формулам:
    $$
    \begin{aligned}
    & p_j^{(k)}=p_j^{(k-1)}+v_j^{\left(2^{k-1}\right)}, \\
    & q_j^{(k)}=2 p_j^{(k)}+q_{j-2^{k-1}}^{(k-1)}+q_{j+2^{k-1}}^{(k-1)}, \\
    & j=2^k, 2 \cdot 2^k, 3 \cdot 2^k, \ldots, N-2^k .
    \end{aligned}
    $$

\end{enumerate}

\section{Обернений хід}


\begin{enumerate}
    \item Задаются значення для $Y_0$ и $Y_N: Y_0=F_0, Y_N=F_N$.

    \item Для кожного фіксованого $k=n, n-1, \ldots, 2$ обчислюється і зберігаються  
    вектори
    $$
    \begin{aligned}
    & v_j^{(0)}=q_j^{(k-1)}+Y_{j-2^{k-1}}+Y_{j+2^{k-1}}, \\
    & j=2^{k-1}, 3 \cdot 2^{k-1}, 5 \cdot 2^{k-1}, \ldots, N-2^{k-1} .
    \end{aligned}
    $$

    Потім при фіксованому $l=1,2, \ldots, 2^{k-1}$ для кожного $j=2^{k-1}$, $3 \cdot 2^{k-1}, 5 \cdot 2^{k-1}, \ldots, N-2^{k-1}$ вирішуються рівняння
    $$
    C_{l, k-1} v_j^{(l)}=v_j^{(l-1)} .
    $$

    В результаті знаходяться вектори $v_j^{\left(2^{k-1}\right)}$. 
    Далі обчислюються $Y_j$ за формулою
    $$
    Y_j=p_j^{(k-1)}+v_j^{\left(2^{k-1}\right)}, \quad j=2^{k-1}, 3 \cdot 2^{k-1}, 5 \cdot 2^{k-1}, \ldots, N-2^{k-1} .
    $$
    
    \item Для $k=1$ вирішується рівняння
    $$
    C Y_j=q_j^{(0)}+Y_{j-1}+Y_{j+1}, \quad j=1,3,5, \ldots, N-1
    $$

\end{enumerate}
