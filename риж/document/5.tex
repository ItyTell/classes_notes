\chapter{Метод роздiлення змiнних (Фур’є)}

Автор розділу - Кирило Кроча

\section{Загальний огляд. Постановка задачі}

Метод розділення змінних, який також часто називають методом Фур'є(така назва випливає з того, що під час знаходження розв'язку використовується його розклад в скінченну суму Фур'є) - це один з багатьох методів розв'язання звичайних диференціальних рівнянь та диференціальних рівнянь в частинних похідних.

Ефективність цього методу залежить від того, наскільки швидко ми зможемо обчислити коефіцієнти Фур' є, а також як швидко зможемо зробити обернене перетворення. Розглянемо приклад - маємо функцію $f(i) \mathrm{i}$ ортонормовану систему $\mu_{k}(i), k=0,1, \ldots, N$, на сітці $\bar{\omega}=\left\{x_{i}=i h, 0 \leq i \leq N\right.$, $h N=l$, з коефіцієнтами перетворення Фур'є, обчисленими за формулою:

$$
\varphi_{k}=\sum_{i=0}^{N} f(i) \mu_{k}(i) h, k=0,1 \ldots, N
$$

Для такого випадку, обчислення усіх $\varphi_{k}$ потребує $(N+1)(N+2)$ операцій множення, і $N(N+1)$ операцій додавання. Для довільної системи $\mu_{k}(i)$ це $\epsilon$ мінімальна кількість необхідних операцій, але існують такі системи, на яких буде достатньо лише $O(N \ln N)$ обчислень.

\section{Розв'язок задачі знаходження власних чисел оператору Лапласа у прямокутнику}

Застосуємо метод розділення змінних для знаходження власних чисел $\lambda_{k}$ та власних функцій $\mu_{k}(i, j)$ для оператору Лапласа.

$$
\Lambda=\Lambda_{1}+\Lambda_{2}, \quad \Lambda_{\alpha} y=y_{\bar{x}_{\alpha} x_{\alpha}}, \quad \alpha=1,2
$$

Припустимо що в прямокутнику $G=\left\{0 \leq x_{\alpha} \leq l_{\alpha}, \alpha=1,2\right\}$ маємо рівномірну прямокутну сітку $\bar{\omega} 3$ кроками $h_{1}, h_{2}$. Позначимо $\omega$ внутрішність, а $\gamma$ межу сітки $\bar{\omega}$.

Найпростіша задача знаходження власних чисел для оператору Лапласа з граничними умовами Діріхле має вигляд: знайти такі значення параметру $\lambda$, що буде існувати нетривіальний розв'язок $y(x)$ наступної задачі:

\begin{equation}
    \begin{cases}
        \Lambda y(x)+\lambda y(x)=0, x \in \omega \\
        y(x)=0, x \in \gamma
    \end{cases}
\end{equation}
    
Ми будемо шукати власні функції (5.1), що відповідають власним числам $\lambda_{k}$, у вигляді:




$$
\mu_{k}(i, j)=\mu_{k_{1}}^{(1)}(i) \mu_{k_{2}}^{(2)}(j), \quad k=\left(k_{1}, k_{2}\right)
$$

Якщо підставимо функцію $\mu_{k}(i, j)$ на місце $y\left(x_{i j}\right)=y(i, j)$ в (1). Оскільки $\Lambda_{1} y(i, j)=\frac{1}{h_{1}^{2}}[y(i+1, j)-2 y(i, j)+y(i-1, j)]$, бачимо, що оператор $\Lambda_{1}$ діє на сіткову функцію, яка залежить від $i$. Аналогічно, оператор $\Lambda_{2}$ діє на функцію, що залежить від $j$. Таким чином, підставляючи вид власних функцій в (1), отримуємо:

\begin{equation*}
\mu_{k_{2}}^{(2)}(j) \Lambda_{1} \mu_{k_{1}}^{(1)}(i)+\mu_{k_{1}}^{(1)}(i) \Lambda_{2} \mu_{k_{2}}^{(2)}(j)+\lambda_{k} \mu_{k_{1}}^{(1)}(i) \mu_{k_{2}}^{(2)}(j)=0 \tag{5.2}
\end{equation*}

Для $1 \leq i \leq N_{1}-1$ i $1 \leq j \leq N_{2}-1$, і також:

$$
\mu_{k_{1}}^{(1)}(0)=\mu_{k_{1}}^{(1)}\left(N_{1}\right)=0, \quad \mu_{k_{2}}^{(2)}(0)=\mu_{k_{2}}^{(2)}\left(N_{2}\right)=0
$$

3 (5.2) отримуємо:

$$
\frac{\Lambda_{1} \mu_{k_{1}}^{(1)}(i)}{\mu_{k_{1}}^{(1)}(i)}=-\frac{\Lambda_{2} \mu_{k_{2}}^{(2)}(j)}{\mu_{k_{2}}^{(2)}(j)}-\lambda_{k}
$$

I оскільки права частина не залежить від $i$, то і ліва частина також не залежить від $i$. Провівши такі самі міркування щодо залежності від $j$, легко зрозуміти, що права і ліва частини - константи. В результаті маємо одновимірні задачі:

$$
\begin{gathered}
\Lambda_{1} \mu_{k_{1}}^{(1)}+\lambda_{k_{1}}^{(1)} \mu_{k_{1}}^{(1)}=0, \quad 1 \leq i \leq N_{1}-1 \\
\mu_{k_{1}}^{(1)}(0)=\mu_{k_{1}}^{(1)}\left(N_{1}\right)=0
\end{gathered}
$$

Ta

$$
\begin{gathered}
\Lambda_{2} \mu_{k_{2}}^{(2)}+\lambda_{k 2}^{(2)} \mu_{k_{2}}^{(2)}=0, \quad 1 \leq j \leq N_{2}-1 \\
\mu_{k_{2}}^{(2)}(0)=\mu_{k_{2}}^{(2)}\left(N_{2}\right)=0
\end{gathered}
$$

Розв'язки цих задач мають вигляд:

$$
\begin{gathered}
\lambda_{k_{\alpha}}^{(\alpha)}=\frac{4}{h_{\alpha}^{2}} \sin ^{2} \frac{k_{\alpha} \pi}{2 N_{\alpha}}=\frac{4}{h_{\alpha}^{2}} \sin ^{2} \frac{k_{\alpha} \pi h_{\alpha}}{2 l_{\alpha}}, \quad k_{\alpha}=1,2, \ldots, N_{\alpha}-1 \\
\mu_{k_{1}}^{(1)}(i)=\sqrt{\frac{2}{l_{1}}} \sin \frac{k_{1} \pi i}{N_{1}}, \quad k_{1}=1,2, \ldots, N_{1}-1 \\
\mu_{k_{2}}^{(2)}(j)=\sqrt{\frac{2}{l_{2}}} \sin \frac{k_{2} \pi j}{N_{2}}, \quad k_{2}=1,2, \ldots, N_{2}-1
\end{gathered}
$$



Таким чином, власні функції та власні вектори для оператору Лапласа були знайдені:

$$
\begin{gathered}
\mu_{k}(i, j)=\mu_{k_{1}}^{(1)}(i) \mu_{k_{2}}^{(2)}(j)=\frac{2}{\sqrt{l_{1} l_{2}}} \sin \frac{k_{1} \pi i}{N_{1}} \sin \frac{k_{2} \pi j}{N_{2}} \\
0 \leq i \leq N_{1}, \quad 0 \leq j \leq N_{2} \\
\lambda_{k}=\lambda_{k_{1}}^{(1)}+\lambda_{k_{2}}^{(2)}=\sum_{\alpha=1}^{2} \frac{4}{h_{\alpha}^{2}} \sin ^{2} \frac{k_{\alpha} \pi h_{\alpha}}{2 l_{\alpha}}
\end{gathered}
$$

Де $k_{\alpha}=1,2, \ldots, N_{\alpha}-1, \alpha=1,2$.

\section{Висновок}

Метод розділення змінних дозволив нам перетворити отриману задачу у декілька одновимірних, для яких було значно легше знайти власні числа і функції, що дозволить спростити розв'язок задачі.