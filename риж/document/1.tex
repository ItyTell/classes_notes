\chapter{Прямi методи розв’язання рiзницевих
рiвнянь}

Автор розділу - Кирило Кроча

\section{Загальний огляд}

Наразі для розв'язання систем лінійних різницевих рівнянь існують два основні типи рівнянь:
\begin{itemize}
    \item Прямі методи;
    \item Ітеративні методи
\end{itemize}    


Загалом, прямі методи частіше орієнтовані на розв'язання вузького класу рівнянь, але дозволяють нам знайти розв'язок з використанням дуже малої частини обчислювальних можливостей.

У цьому розділі ми розглянемо загальні ідеї прямих методів, що використовуються для розв'язання різницевих рівнянь. Для цього розглянемо важливий клас різницевих рівнянь - лінійне рівняння з сталими коефіцієнтами.

\section{Загальна теорія лінійних різницевих рівнянь}

Нашою метою є знаходження лінійно незалежних розв'язків рівняння порядку $m$ :

$$
a_{m} y(i+m)+a_{m-1} y(i+m-1)+\cdots+a_{0} y(i)=0
$$

Ми будемо шукати розв'язки цього рівняння у формі $v(i)=q^{i}$. Підставляючи такий вигляд у рівняння, отримуємо:

$$
q^{i}\left(a_{m} q^{m}+\cdots+a_{0}\right)=0
$$

Але оскільки ми шукаємо ненульовий розв' язок, можна \\поділити на $q^{i}$ :

$$
\left(a_{m} q^{m}+\cdots+a_{0}\right)=0
$$

Це рівняння називають характеристичним рівнянням, його корені можуть бути простими або кратними.

Припустимо, що вони прості, тоді можна показати що функції \\ $v_{1}(i)=$ $q_{1}^{i}, \ldots, v_{m}^{i}=q_{m}^{i}$ є лінійно незалежними розв'язками початкового рівняння. 3 цього випливає, що загальний розв'язок однорідного рівняння можна записати у вигляді:

$$
y(i)=c_{1} q_{1}^{i}+\cdots+c_{m} q_{m}^{i}
$$

Де $c_{1}, c_{2}, \ldots, c_{m}$ - це довільні константи. Для кратних коренів можна отримати схожі результати, розв'язок буде мати наступний вид:


$$
y(i)=\sum_{l=1}^{s} \sum_{n=0}^{n_{l}-1} c_{n}^{(l)} j^{n} q_{l}^{j}
$$

Для знаходження розв'язку неоднорідного рівняння як суму загального розв'язку однорідного і частинного розв'язку неоднорідного рівнянь, необхідно знайти частинний розв'язок неоднорідного рівняння. Для рівняння другого порядку його можна записати як:

$$
\bar{y}(n)=\sum_{k=n_{0}}^{n-2} \frac{q_{2}^{n-k-1}-q_{1}^{n-k-1}}{q_{2}-q_{1}} \cdot \frac{f(k)}{a_{2}}
$$

\section{Висновок}

Ми зрозуміли, що прямі методи грають важливу роль у вирішенні різницевих рівнянь, і хоча вони застосовні не до усіх видів рівнянь, іноді вони дозволяють значно скоротити час обчислень. Також ми розглянули деякі елементи теорії розв'язання лінійних різницевих рівнянь, що допоможе застосовувати прямі методи на практиці.