
\section{Перша задача}

\begin{tcolorbox}[title=Задача 1]
    \begin{center}
        Розрахувати $\int_{0}^{1}(3x+1)^2dx$ за формулою лівих прямокутників з трьома вузлами. 
        Оцінити точність за правилом Рунге (взяти крок між вузлами та вдвічі більший). Який крок треба взяти, щоб точність $\varepsilon=0.05$ була гарантована.(використати апріорну оцінку) 
    \end{center}
\end{tcolorbox}



\begin{center}
    Запишемо інтеграл, як сумму інтегралів по вузлам:

    $$\int_{0}^{1}(3x+1)^2dx = \int_{0}^{\frac{1}{4}}(3x+1)^2dx + 
    \int_{\frac{1}{4}}^{\frac{1}{2}}(3x+1)^2dx + 
    \int_{\frac{1}{2}}^{\frac{3}{4}}(3x+1)^2dx + 
    \int_{\frac{3}{4}}^{1}(3x+1)^2dx$$

    для кожного інтегарлу використаємо формулу лівих прямокутників \
    $\int_{a}^{b}f(x)dx = (b-a)f(a)$:
    $$ 
    \begin{cases}
        \int_{0}^{\frac{1}{4}}(3x+1)^2dx = \frac{1}{4}\\
        \int_{\frac{1}{4}}^{\frac{1}{2}}(3x+1)^2dx = \frac{1}{4} \frac{49}{16} \\
        \int_{\frac{1}{2}}^{\frac{3}{4}}(3x+1)^2dx = \frac{1}{4} \frac{25}{4}\\
        \int_{\frac{3}{4}}^{1}(3x+1)^2dx = \frac{1}{4} \frac{169}{16}
    \end{cases}
    $$

    $$ 
    \int_{0}^{1}(3x+1)^2dx = \frac{1}{4 * 16}(1 + 49 + 100 + 169) = \frac{319}{64} 
    $$

    Повторимо процедуру для кроку $\frac{1}{2}$

    $$\int_{0}^{1}(3x+1)^2dx = \int_{0}^{\frac{1}{2}}(3x+1)^2dx + 
    \int_{\frac{1}{2}}^{1}(3x+1)^2dx = \frac{1}{2}(1 + \frac{25}{4}) = \frac{29}{8}$$

    Оцінимо точність за правилом Рунге:

    $$ \varepsilon_0 = \frac{|y_{\frac{1}{4}} - y_{\frac{1}{2}}|}{3} = 
    \frac{319 -29 * 8}{64 * 3}  = \frac{29}{64}$$

    Формула апріорної оцінки $\varepsilon \le \frac{M_1 (b-a)}{2n}$, 
    у нас похідна функції зростаюча, тобто $M_1 = f'(b)$, отже

    $$\varepsilon \le \frac{24}{2n} \rightarrow \text{якщо} \varepsilon = 0.05, 
    n = \frac{12}{0.05} = 240$$
\end{center}



