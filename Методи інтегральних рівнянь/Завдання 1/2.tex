\section{задача}


\begin{tcolorbox}[title=Умова]
    Дослідіть на диференційовність функцію
    $$
    f(x)=\frac{1}{2}(A x, x)-(b, x),
    $$

    де $A: \mathbb{R}^n \rightarrow \mathbb{R}^n$ - самоспряжений лінійний оператор, $b \in \mathbb{R}^n$.
\end{tcolorbox}


$$f'(x) = \lim_{h \to 0} \frac{f(x+h) - f(x)}{h} = 
\lim_{h \to 0} \frac{
    \frac{1}{2}(A(x+h), x+h) - (b, x+h) - \frac{1}{2}(Ax, x) + (b, x)}{h}=
$$
$$
= \lim_{h \to 0} \frac{
    \frac{1}{2}(Ax, x) + \frac{1}{2}(Ah, x) + \frac{1}{2}(Ax, h) + \frac{1}{2}(Ah, h) - (b, h) - \frac{1}{2}(Ax, x)}{h}=
$$
$$
= \lim_{h \to 0} \frac{
    \frac{1}{2}(Ah, x) + \frac{1}{2}(Ax, h) + \frac{1}{2}(Ah, h) - (b, h)}{h}=
    \lim_{h \to 0}\frac{O(h)}{h} = O(1)
$$

Скалярний добуток неперервний і $f'(x) = O(1)$ - отже $f(x)$ диференційовна.