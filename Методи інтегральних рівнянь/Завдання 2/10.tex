
\section{задача}
\begin{tcolorbox}[title=Умова]
Нехай $x$ - розв'язок задачі Коші
$$
\left\{\begin{array}{l}
\frac{d}{d t} x=-A x+b, \\
x(0)=x_0
\end{array}\right.
$$

де $b, x_0 \in \mathbb{R}^n, A$ - симетрична додатньо визначена матриця $n \times n$. Що можна сказати про $\lim _{t \rightarrow+\infty} x(t)$ ? Відповідь обгрунтуйте.

\end{tcolorbox}


Те саме, що і в минулом завданні тільки тепер $A \neq 0$. Тоді $x(t)$ буде розбігатися на нескінченності, якщо взяти її одиничною, а вектор $b$ взяти оберненого знака до 
$x_0$ або довільного, якщо той нуль. І буде прямувати до $x_0$, якщо вектор $b$ взяти 
його рівним $x_0$. Тож однозначної відповіді немає.