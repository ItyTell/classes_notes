\section{задача}

\begin{tcolorbox}[title=Умова]
    Нехай $f : E \rightarrow \mathbb{R}$  $\mu$-сильно опукла $L$-гладка функцiя. Розглянемо
    $$
    \begin{cases}
        \ddot{x} + 2 \sqrt{\mu} \dot{x} = - \nabla f(x) \\
        x(0) = x_0, \quad \dot{x}(0) = 0, \quad \mu > 0
    \end{cases}
    $$
    Доведіть, що 
    $$ f(x(t)) - f_* \le 2 e^{-\sqrt{\mu}t}(f(x_0) - f_*) \quad \forall t > 0 $$
\end{tcolorbox}


$$
\ddot{x} + 2 \sqrt{\mu} \dot{x} = 0 
$$

$$x(t) = e^{-2\sqrt{\mu} t} + C_1$$

З минулої задачі маємо, що


$$x^* \in argmin_{y \in E} \{f(y) + x(y)\} \quad
\Leftrightarrow\quad x^* = prox_{\lambda x}(x^* + \lambda \nabla f(x)) $$

$$x^* = prox_{\lambda x}(x^* + \lambda \nabla f(x^*))= argmin_{y \in E}
\{\lambda x(y) + \frac{1}{2} \|x^* + \lambda \nabla f(x^*) - y\|^2 \}$$

$f_* = f_{min}$
$x^*(0) = x_0, \quad \dot{x}^*(0) = 0$

$$\frac{1}{2\sqrt{\mu}}\dot{x} + x = f(x) $$

Не знаю, як далі, заплутався.