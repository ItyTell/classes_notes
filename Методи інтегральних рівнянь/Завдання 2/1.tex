\section{задача}

\begin{tcolorbox}[title=Умова]
    Доведiть, що для гладкої задачi опуклого програмування $f \rightarrow \min\limits_C$ має мiсце:

    $$ f(x) = \min\limits_C f \Leftrightarrow x \in C \quad 
    \text{та} \quad (\nabla f(x),  y - x) \geq 0, \quad \forall y \in C$$
\end{tcolorbox}

Для опуклої функції виконуеться нерівність (минула дз)
$$
f(y) \geq f(x)+(\nabla f(x), y-x),
$$

Якщо $\forall y \in C$ маємо $(\nabla f(x), y-x) \geq 0$ то
$$
f(y) \geq f(x)+(\nabla f(x), y-x) \geq f(x) \quad \forall y \in C,
$$

тобто $f(x)=\min\limits_C f$.


Тепер навпаки(зпарва наліво), запишемо тепер наближення першого порядку для $f$ в $x$ :
$$
f(x) \leq f(y)=f(x)+(\nabla f(x), y-x)+o(\|y-x\|), \quad \forall y \in C .
$$

Оскільки допустима множина $C$ опукла, то разом із точками $x$ та $y$ до неї входять довільні їхні опуклі комбінації, тобто точки вигляду $(1-\lambda) x+\lambda y$, де $\lambda \in(0,1)$. Підставляемо ї у останно нерівність:
$$
0 \leq \lambda(\nabla f(x), y-x)+o(\lambda\|y-x\|), \quad \forall y \in C
$$

Спрямовуючи $\lambda \rightarrow 0$ бачимо, що знак правої частини визначається перпим доданком (властивості о-малого), а тому маємо нерівність
$$
0 \leq(\nabla f(x), y-x), \quad \forall y \in C .
$$