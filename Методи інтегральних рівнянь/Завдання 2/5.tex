\section{задача}

\begin{tcolorbox}[title=Умова]
    Нехай $f : E \rightarrow R $ опукла та $L$-гладка вiдносно норми 
    $\|  \|_2$ функцiя, $g : E \rightarrow \mathbb{R} \cup {+\infty} $ 
    власна замкнена та опукла функцiя. Доведiть, що
    $$ x = prox_{\lambda g}(x - \lambda \nabla f(x)) \quad 
    \Leftrightarrow \quad x \in argmin_{y \in E} \{f(y) + g(y)\} $$
\end{tcolorbox}


Зліва направо впипливає з задачі 3 і з градієнтного методу. 
Доведемо зправа наліво. Нехай $x \in argmin_{y \in E} \{f(y) + g(y)\}$. 
Тоді знову ж таки з задачі 3 і з того, що градієнтний метод збігається, маємо доведення.
Не знаю, як по іншому записати це доведення, сподіваюсь пройде.